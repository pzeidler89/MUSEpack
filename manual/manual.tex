\documentclass{pz_thesis}

\pagestyle{fancy}
\fancyhf{}
\fancyhead[LE,RO]{\thepage}
\fancyhead[LO]{\leftmark}
\fancyhead[RE]{\rightmark}


\usepackage{XCharter} % Use the Bitstream Charter font
\usepackage{natbib}
\setcitestyle{aysep={}} 
\usepackage{breakurl}
\usepackage{breakcites}
\usepackage{parskip}
\usepackage{emptypage}

\usepackage{float}
\definecolor{Red}{rgb}{1.0, 0.0, 0.0}

\setlength{\parskip}{5pt plus 1pt minus 1pt}

\title{MUSEpack}

\author{Peter Zeidler}

\date{\today}

\begin{document}
\maketitle
\tableofcontents
\listoffigures
\listoftables
\newpage
\thispagestyle{empty}
\mbox{    }

\section{MUSEreduce}

The MUSEreduce module is meant to streamline the process of using the standard reduction pipeline provided by ESO\footnote{\url{https://www.eso.org/sci/software/pipelines/muse/}} \citep{Weilbacher_12,Weilbacher_14} to reduce the data. This pipeline is based on the ESO Reflex environment (ESORex) for automated data reduction work flows for astronomy \citep{Freudling_13}. The standard pipeline need to be installed in order for the package to work.

The directory, in which the data reduction takes place should contain a directory \texttt{raw}, where all the archive raw data of the obsevations are placed. One calls the module with the *.json filename as argument where all the options are set. The default file is found in the directory and is called "config.json" with the following options:

\subsubsection{GLOBAL}

\begin{itemize}
	\item[\textbf{withrvcorr}] bool, true if barycentric RV correction should be performed, default: true 
\end{itemize}

\subsubsection{CALIBRATION}

\subsubsection{SCIBASIC}

\subsubsection{STD\_FLUX}

\subsubsection{SKY}

\subsubsection{SCI\_POST}

\subsubsection{EXP\_COMBINE}

%    withrvcorr = True #Set True if you want sky subtraction and heliocentric correction (recommended)
%using_ESO_calibration = False #Set True if you want to use the ESO provided calibration files (recommended)
%auto_sort_data = True #Set True if the script shall sort the raw data automatically (recommended)
%using_specific_exposure_time = False #Set specific exposure time if needed, otherwise set False
%dithering_multiple_OBs = False # Set if individual dither positions are distributed via multiple OBs
%n_CPU=24 #Set the number of Threads used to reduce the data manually. Default and recommended is 24.
%
%OB_list=np.array(['long_1a','long_1b','long_1c'])#,'long_2b','long_2c']) #Name of the OB folder
%
%dithername='OB2' #Name of the pointing, for which multiple OBs are used to dither
%
%manual_rootpath='/astro/zeidler/NGC346/' #set if you don't want to use the location of this script
\bibliographystyle{pz_bib}



{\footnotesize\bibliography{bibliography}}



\end{document}